%%%%%%%%%%%%%%%%%%%%%%% file template.tex %%%%%%%%%%%%%%%%%%%%%%%%%
%
% This is a general template file for the LaTeX package SVJour3
% for Springer journals.          Springer Heidelberg 2010/09/16
%
% Copy it to a new file with a new name and use it as the basis
% for your article. Delete % signs as needed.
%
% This template includes a few options for different layouts and
% content for various journals. Please consult a previous issue of
% your journal as needed.
%
%%%%%%%%%%%%%%%%%%%%%%%%%%%%%%%%%%%%%%%%%%%%%%%%%%%%%%%%%%%%%%%%%%%
%
% First comes an example EPS file -- just ignore it and
% proceed on the \documentclass line
% your LaTeX will extract the file if required
\begin{filecontents*}{example.eps}
%!PS-Adobe-3.0 EPSF-3.0
%%BoundingBox: 19 19 221 221
%%CreationDate: Mon Sep 29 1997
%%Creator: programmed by hand (JK)
%%EndComments
gsave
newpath
  20 20 moveto
  20 220 lineto
  220 220 lineto
  220 20 lineto
closepath
2 setlinewidth
gsave
  .4 setgray fill
grestore
stroke
grestore
\end{filecontents*}
%
\RequirePackage{fix-cm}
%
%\documentclass{svjour3}                     % onecolumn (standard format)
%\documentclass[smallcondensed]{svjour3}     % onecolumn (ditto)
\documentclass[smallextended,twocolumn]{svjour3}       % onecolumn (second format)
%\documentclass[twocolumn]{svjour3}          % twocolumn
%
\tolerance=1
\emergencystretch=\maxdimen
\hyphenpenalty=10000
\hbadness=10000

\smartqed  % flush right qed marks, e.g. at end of proof
%
\usepackage{graphicx}
%
% \usepackage{mathptmx}      % use Times fonts if available on your TeX system
%
% insert here the call for the packages your document requires
%\usepackage{latexsym}
% etc.
%
% please place your own definitions here and don't use \def but
% \newcommand{}{}
%
% Insert the name of "your journal" with
% \journalname{myjournal}
%
\begin{document}

\title{Control Menu based Spatial Awareness%\thanks{Grants or other notes
%about the article that should go on the front page should be
%placed here. General acknowledgments should be placed at the end of the article.}
}
\subtitle{}

%\titlerunning{Short form of title}        % if too long for running head

\author{Yuan Shuai         \and
        Sun Minghui %etc.
}

%\authorrunning{Short form of author list} % if too long for running head

\institute{F. Author \at
              first address \\
              Tel.: +123-45-678910\\
              Fax: +123-45-678910\\
              \email{fauthor@example.com}           %  \\
%             \emph{Present address:} of F. Author  %  if needed
           \and
           S. Author \at
              second address
}

\date{Received: date / Accepted: date}
% The correct dates will be entered by the editor


\maketitle

\begin{abstract}
Today,people's life cannot leave the electric equipments,such as mobile phone,ipad and other equipments that need people to control.There have been so much inputs,such as mouses,styluses and figures.Traditional ways of interection mainly provide x-y position to allow users to control the menu,but they provide z position rarely.Most of inputs are based on touch screens on the equipment or buttons on the control table.The spatial awareness has been always ignored.we will discuss a new different input way base on the spatial awareness.We divide the space in front of users into some small cube space (SCS).People can click the shortcut key on the contral menu by select the the specific SCS in the front of them using their hands, with full or partial visual feedback.In this paper,we design the experiment to invest human's ability to select a SCS exactly using this sense. And the experiment also considers two selection methods to c users confirm their selection once the the SCS acquired by their hand,we also give some questionnaires to participations to collect user feedback informations.\emph
\keywords{Perception of space \and Control menus \and Human computer interection}
% \PACS{PACS code1 \and PACS code2 \and more}
% \subclass{MSC code1 \and MSC code2 \and more}
\end{abstract}

\section{Introduction}
\label{intro}
Traditional HCI has been designed to two degree-of-freedom mapping the x-y position that mouse, styluse or figure always provide.In addation to these inputs, there also have rockers and wheels provide x-y position sameily.So much papers discuss these inputs and provide much improvement programs based on these inpus.These inputs has been widespred used in our daily life. 
But in some situtions, no mater how to improve these inputs,they have limitations,for example, when you use AR device, it's very inconvenient to use traditional inputs,expecially to use immersing AR  device which user can hardly get the outside informations but easily get the space information.Expecialy with the virtual visual feedback in the AR device,this Menu Contral Function can perform better.And in the file of large screen contral,using human's spatial awareness can assist the visual impairment peopel to make contral the device easily,like large screen.
\\
\\
If we want to use the human spatial awareness to make menu contral like click the shortcut key by select SCS,we should know how much can human know about the space around themself.In this paper, we design a experiment to invest users' spatial awareness.Question that need to be answered include:how much discrete layers the space in front of user can be divided in to SCS in vertical and horizontal directions, what mechanisms can be used to confirm the users' selection, and what is the impact of visual feedback, how much difference between right-handed and left-handed when they are supposed to select a SCS.
\\
\\
First of all,we review some relevant reserachers' works.Then we'll present our experiment to invest humans' ability to select the the specific SCS in the front of them using their hands, with full or partial visual feedback.Our experiment also conside different techniques for confirming users' selection once the SCS is acquired. 
\section{Preview Work}
\label{sec:1}
Frank Chun Yat Li et al explore the kinesthetic memory and spatial awareness used in the mobile device.they 
\subsection{Subsection title}
\label{sec:2}
as required. Don't forget to give each section
and subsection a unique label (see Sect.~\ref{sec:1}).
\paragraph{Paragraph headings} Use paragraph headings as needed.

\begin{equation}
a^2+b^2=c^2
\end{equation}

% For one-column wide figures use
\begin{figure}
% Use the relevant command to insert your figure file.
% For example, with the graphicx package use
  \includegraphics{example.eps}
% figure caption is below the figure
\caption{Please write your figure caption here}
\label{fig:1}       % Give a unique label
\end{figure}
%
% For two-column wide figures use
\begin{figure*}
% Use the relevant command to insert your figure file.
% For example, with the graphicx package use
  \includegraphics[width=0.75\textwidth]{example.eps}
% figure caption is below the figure
\caption{Please write your figure caption here}
\label{fig:2}       % Give a unique label
\end{figure*}
%
% For tables use
\begin{table}
% table caption is above the table
\caption{Please write your table caption here}
\label{tab:1}       % Give a unique label
% For LaTeX tables use
\begin{tabular}{lll}
\hline\noalign{\smallskip}
first & second & third  \\
\noalign{\smallskip}\hline\noalign{\smallskip}
number & number & number \\
number & number & number \\
\noalign{\smallskip}\hline
\end{tabular}
\end{table}


%\begin{acknowledgements}
%If you'd like to thank anyone, place your comments here
%and remove the percent signs.
%\end{acknowledgements}

% BibTeX users please use one of
%\bibliographystyle{spbasic}      % basic style, author-year citations
%\bibliographystyle{spmpsci}      % mathematics and physical sciences
%\bibliographystyle{spphys}       % APS-like style for physics
%\bibliography{}   % name your BibTeX data base

% Non-BibTeX users please use
\begin{thebibliography}{}
%
% and use \bibitem to create references. Consult the Instructions
% for authors for reference list style.
%
\bibitem{RefJ}
% Format for Journal Reference
Gonzalo Ramos,\emph{et al}. Pressure Widgets,  \emph{ACM CHI 2004, Volume 6, Number 1} (2004)
% Format for books
\bibitem{RefB}
Author, Book title, page numbers. Publisher, place (year)
% etc
\end{thebibliography}

\end{document}
% end of file template.tex

